%% Compilation :
%% pdflatex "rapport(term).tex" ; makeindex "rapport(term).idx" ; bibtex "rapport(term)" ; pdflatex "rapport(term).tex" ; pdflatex "rapport(term).tex"
%%
\documentclass[10pt,a4paper]{report}
\usepackage[titles]{mypack}

\docinfo {
\small{
	\begin{flushleft}
		\dateFirstEdit{12 janvier 2009} \\
		\dateLastEdit{2 f�vrier 2009} \\
		\versionApp{V0} \\
		\docState{Termin�} \\
		\reference{https://services.emi.u-bordeaux1.fr/projet/viewvc/botinsl2009/trunk/Rapport/rapport-28term-29.pdf?view=log}
	\end{flushleft}
	}
}
%----------------------------------------------------------

%--------- informations visible sur le titre -----------
\title{Bot in Second Life}
\subtitle{Individu autonome pour la recherche d'informations sur Second Life}
\author{
	\student{Besse} {Julien} \vskip 0.2cm
	\student{Blondeau} {Olivier} \vskip 0.2cm
	\student{Delahaye} {Thomas} \vskip 0.2cm 
	\student{Glacet} {Christian}  \vskip 0.2cm
}
\teacher{Client : Philipe Narbel \\ Charg� de TD : Pascal Desbarat}
%---------------------------------------------------------

% Saut entre les "paragraphes"
\setlength\parskip{0.1cm}
% Profondeur de la num�rotation
\setcounter{secnumdepth}{3}
% Cr�ation de l'index
\makeindex

%%%%%%%%%%%%%%%%%%%%%DOCUMENT%%%%%%%%%%%%%%%%%
\begin{document}

% Met les commentaire en marge � gauche (au lieu de droite par d�fault)
\reversemarginpar{}
%------ Titre ------
\maketitle

%---- R�sum� -------
\begin{abstract}
Ce projet consiste en la cr�ation d'un bot (entit� virtuelle autonome) qui adoptera un comportement humain afin de rechercher des informations dans le Monde de Second Life. Pour se faire, le bot devra se d�placer, discuter et faire des choix de mani�re autonome. L'utilisateur final de ce projet devra juste donner des caract�ristiques personnelles au bot, pour le rendre plus humain, ainsi que l'objet de sa recherche. Le bot sera alors lanc� dans ce monde virtuel pour commencer son investigation et trouver une r�ponse � sa recherche. Le bot fournira un rapport sur sa recherche et sera aussi capable d'am�liorer sa base de connaissances gr�ce � ses diff�rentes recherches. Ce projet sera r�alis� � l'aide du projet open-source libSL qui nous a fourni les m�thodes pour faire interagir un bot avec le monde de Second Life.

% Mieux faire quand on aura le temps : 
%\hugejump 
%\begin{center}
%	\textbf{Remerciements}
%	\jump
%\end{center}
%Ce compte rendu a �t� �crit en \LaTeX ~\xspace � l'aide un tutoriel de Josselin Noirel~\cite{JNLatex}.

\end{abstract}

%---- Sommaire ------
\setcounter{tocdepth}{2} %-profondeur du som.
\tableofcontents
%--------------------

%--- Chapitres ------
% D�finition du domaine d'application
%-- D�finitions n�cessaires � la compr�hension du sujet --
\chapter{Domaine d'application}

%------------------------------------------------------------
\section{Objectifs du projet de programmation}
Le but (scolaire) de ce projet est de suivre une d�marche qualit� assez simple, cela dans le but de nous former pour une meilleure insertion en entreprise. 
Nous devrons donc tout au long de ce projet �diter des documents li�s � la conduite de projet (analyse de besoins, journal de bord, ...). Le but principal de ce projet est donc de produire des documents clairs et pr�cis sur l'avancement du projet, la d�marche adopt�e, les probl�mes rencontr�s et les solutions apport�es. 
L'int�r�t de la chose �tant notamment d'obtenir un projet qui quelque soit son �tat d'avancement puisse �tre repris par une autre �quipe de d�veloppement sans perdre trop de temps. 
Ou encore de permettre � une personne ext�rieure d'int�grer l'�quipe facilement.
%----------------------------------------------------------

%----------------------------------------------------------
\section{La recherche d'informations}
La recherche d'informations est la science de trouver des informations dans un document de tous type. L'important pour une recherche d'informations est de cibler la demande de l'utilisateur afin de mieux trouver le document. Pour illustrer ce point, quand un utilisateur effectue une recherche via le site Google, il a int�r�t � cibler ses mots cl�s pour trouver une r�ponse pr�cise.

Pour se faire, il existe plusieurs m�thodes dans la recherche d'informations :

\subsection{Le mod�le de l'information Retrieval}
Ce mod�le consiste � placer un \keyword{agent interm�diaire} entre l'usager et la base de documents. 
Cet agent est charg� de chercher la bonne source d'informations par rapport aux crit�res de l'usager. 
Ce mod�le s'applique � notre exemple pr�c�dent: Google. 
Le site fait le lien entre l'usager (� travers ses mots cl�s) et la base de donn�es des sites r�f�renc�s par Google.

\subsection{Les repr�sentations bas�es sur l'exploration}
Cette repr�sentation permet � un usager d'effectuer des recherches quand son besoin de d�part est assez flou. Il commence avec des documents tr�s g�n�raux et en farfouillant, affine sa recherche pour arriver � des documents plus pr�cis. 
Pour mettre en parall�le avec Internet, l'usager commence sur un site g�n�ral sur le domaine puis ce site va l'entra�ner sur d'autres liens. 
Apr�s quelques visites de sites, l'usager va comprendre le sujet, affiner ses besoins et choisir les mots cl�s qui lui permettront de faire �ventuellement une recherche de mod�le Retrieval.

\subsection{La construction d'un espace de recherche}
Pour construire un syst�me de recherche d'informations de ce type, l'utilisateur doit d�finir une repr�sentation de d�part, la repr�sentation finale et les actions possibles entre ces �tapes. 
Ce syst�me est assez cibl� au final. Il permet par exemple de chercher comment r�soudre le probl�me de la tour d'Hano�.

Voici quelques exemples de syst�mes de recherche. 
Ces syst�mes basent leur recherche sur un document, c'est � dire dans n'importe quel support o� se trouve de l'information. 
On peut consid�rer que le monde virtuel de Second Life est un document � part enti�re. 
En lisant les principes du bot, vous comprendrez facilement que notre bot sera consid�r� comme un agent interm�diaire � la recherche faisant le pont entre l'utilisateur qui recherche une information et le monde de Second life, l� o� se trouvent les r�ponses (mod�le Retrieval). 
Le mod�le se basant sur l'exploration pourra �tre aussi utile � notre recherche d'informations, notre bot fouinera un peu partout sur le monde de Second Life dans le but d'am�liorer sa base de connaissance sur un sujet. 
%----------------------------------------------------------

%----------------------------------------------------------
\section{\SL (SL)}
Second life est un univers virtuel en 3D,qui a d�but� en 2003. Cet univers est accessible par Internet mais Second life n'est pas un jeu massivement multi-joueur classique. En effet, ici pas de qu�tes ou de buts pr�d�finis, chaque joueur est libre de parcourir le monde comme bon lui semble. Certains y chercheront un c�t� social en profitant des nombreuses possibilit�s de discutions entre joueurs, d'autres privil�gierons une d�marche �conomique. D'o� le titre du jeu, "Seconde Vie", chaque joueur se cr�e une identit� propre (\keyword{avatar}) qu'elle soit proche de la r�alit� ou invent�e de toute pi�ce.Ensuite, chaque joueur pourra apporter sa pierre � cet univers. Contrairement � beaucoup de jeu en ligne, Second Life donne la possibilit� aux joueurs d'apporter des modifications � cet univers, en donnant la possibilit� d'acheter des parcelles, y construire des �difices et fabriquer des objets.

Second Life est accessible gratuitement ce qui a permis d'attirer des millions d'avatars � travers le monde. 14 millions de comptes ont �t� cr�es depuis 2003, m�me si en r�alit� 200 000 � 300 000 comptes seraient jou�s r�guli�rement et environ 50 000 joueurs quotidien. Cet engouement populaire et le c�t� commercial libre a entra�n� un v�ritable d�cha�nement �conomique. Un joueur peut acqu�rir de la monnaie virtuelle (Linden Dollar) contre de vrais dollars, mais il existe beaucoup d'autres moyens pour gagner de l'argent. A l'image de notre soci�t�, des parcelles de terrain qui selon leur c�te peuvent �tre vendu plus ou moins cher. Plus il y a d'avatars passant sur une parcelle , plus la valeur de celle-ci augmentera. Beaucoup de compagnies internationales(Adidas, Dior, Toyota...) ont donc investis dans cet univers pour en faire une vitrine virtuelle. Les joueurs peuvent cr�er des objets qu'ils peuvent vendre. Les objets sont brevet�s, ce qui garantit au joueur de pouvoir profiter sereinement de son commerce. L'univers est donc architectur� et d�cor� par les joueurs. Ce monde conna�t du succ�s car il permet aux joueurs de donner libre cours � leur imagination que ce soit en termes d'objectifs personnels ou de cr�ativit�. 
%----------------------------------------------------------

%----------------------------------------------------------
\section{\LL}
La soci�t� qui a cr�� \SL fut fond�e en 1999 par Phillip Rosedate. Jusqu'en Mars , il occupait la place de CEO (Chief Execute Officer) de cette soci�t�. A partir de cette date, il a voulu se concentrer plus particuli�rement sur l'aspect strat�gique et sur l'orientation de l'entreprise. D�barrass� de cette charge administrative, il est n�anmoins ''cha�man of the board'' (Pr�sident du Conseil d'Administration). Au d�but de l'ann�e 2007, Linden Lab d�cide de passer une partie du code de \SL Viewer sous la d�signation Open Source.

Nous ferons donc un bref r�sum� de ce qui a pu pousser Linden Lab � changer sa philosophie sur le sujet, ainsi que les risques que cela implique.

Presque quatre ans apr�s son lancement, le mouvement Second Life continue de prendre de l'ampleur. 
Le client du monde parall�le de Linden Lab vient en effet de passer sous licence GPL, ce qui rend cette partie enti�rement recopiable et r�utilisable par toute personne souhaitant cr�er son propre client en gardant comme base le coeur de \SL ; 
Le code est disponible sur le site officiel de \SL : \footnote{SL \url{www.secondlife.com}}. 
D'apr�s un des sites officiels \footnote{\SL faq : \url{http://secondlifegrid.net/technology-programs/virtual-world-open-source/faq}}, nous pourrons comprendre quels sont les int�rets pour \SL de laisser du code "ouvert" : L'int�r�t principal de cette nouveaut� est de permettre aux d�veloppeurs d'acc�der � la version compl�te de \SL en y ajoutant du contenu personnalis� tels que des programmes ou des scripts. 
En laissant ce code Open Source, les cr�ateurs du jeu souhaitaient avant tout rapprocher la communaut� de joueurs et les programmeurs de \SL. 
Ceci permettra aux d�veloppeurs d'obtenir un point de vue pr�cis venant des utilisateurs envers les modifications apport�es tandis que ces derniers seront mieux inform�s des nouveaut�s et sauront donc comment am�liorer leur (le) monde virtuel. 
Le second int�r�t sera d'engendrer une plus forte cr�ativit� de la part des joueurs. Le fait qu'une partie du code soit pass� en Open Source favorise amplement la cr�ation d'outils ou de programmes permettant de cr�er divers objets sur \SL. 
La communaut� de joueur pourra alors cr�er plus facilement des objets au fur et � mesure que le nombre d'outils grandira ! 
Les concepteurs de ce monde virtuel pensent m�me � passer une plus grosse partie de leur code d'application "ouvert" s'ils consid�rent la premi�re exp�rience comme r�ussie, permettant ainsi aux joueurs d'avoir une meilleure connaissance de ce monde. 
Ceci entra�nerait, normalement, une croissance dans le domaine de la cr�ation d'outils de toutes sortes pour \SL. 
Enfin, pour terminer, les cr�ateurs estiment que le fait de laisser libre une partie de leur code permettrait de d�velopper plus de fonctionnalit�s, de corriger plus rapidement les bugs exploitables et ainsi am�liorer la s�curit� du monde virtuel beaucoup plus rapidement, et ceci par le fait que les utilisateurs contribuent � la d�couverte et r�solution d'un de ces points.

Nous noterons tout de m�me que durant cette ann�e de 2007, \SL va vivre une ann�e de tr�s forte m�diatisation dans le monde. Depuis, le studio ne fait plus grand bruit et la communaut� de fid�les reste toujours pr�sente. 
Nous ne pourrions pas affirmer que l'Open Source � contribu� � cette "mode" mais le supposer n'est pas non plus impensable.

Comme nous venons de le voir, passer une partie de leur code en Open Source permet aux cr�ateurs d'envisager une quantit� importante de points positifs. Mais tout cela � un prix ; 
laisser une portion de code � la port� de toute personne peut se r�v�ler dangereux. 
Par exemple, plus la quantit� de code partag�e est importante, plus on a de risque de se divulguer des informations importantes. Divulguer une partie du code permet aux "utilisateurs" de cette partie de rep�rer d'�ventuelles \index{failles de s�curit�}failles de s�curit�. 
Pour parer � cette possibilit�, Linden Lab avoue ne pas avoir mis au point un syst�me de d�fense visant � cloisonner le code. 
Ils pr�f�rent laisser au contraire un maximum de transparence aux personnes souhaitant r�utiliser leur code. Mais ils restent sceptiques quant � la robustesses de leur code, en s'exprimant de mani�re assez �tonnante sur un de leur sites officiels (voir lien ci-dessus) :
\myquote
{[...] if someone is determined to break in, has access to the right resources and skills, and enough time - they may well succeed.}
{FAQ \SL \footnote{\SL faq \url{http://secondlifegrid.net/technology-programs/virtual-world-open-source/faq}}}
Le plus surprenant dans cette d�claration officielle est la r�v�lation au grand jour de la vuln�rabilit� de leur code ! La citation exprime clairement l'id�e que si une personne suffisamment dou�e dans le hacking souhaite r�ellement d�molir le \SL Grid, elle y arrivera certainement... 
La d�claration suivante est toute aussi �tonnante si l'on se souvient que \SL est un jeu � but lucratif : \myquote{[...] the Second Life Grid is as much at risk as any other online business.}
{FAQ \SL}.

Pour finir cette pr�sentation, nous dirons que cette d�cision n'a pas �t� prise � la l�g�re comme le d�montre la dur�e des tests effectu�s par l'�quipe de d�veloppement de \SL : 
\myquote{However, we're not solely relying on our switch to a more robust development model to ensure greater security for the Second Life viewer. We've spent months performing a security audit of our design and our source code to reduce the risk that the increased attention we'll be receiving as we release this won't result in a spate of vulnerability discoveries.}{FAQ \SL}. Tout ceci dans le but de rendre plus stable l'application pour laisser le moins de possibilit�s exploitables � une personne malveillante...
%----------------------------------------------------------

%----------------------------------------------------------
\section{\index{Individu autonome}Individu autonome}
Pour faire des recherches sur \SL, il va falloir concevoir un individu capable d'effectuer des recherches sans utilisateur aux commandes. Pour se faire, il faut concevoir un individu qui pourra parler (pour pouvoir se renseigner) et se d�placer (pour pouvoir aller aux gens porteurs d'informations).

\subsection{L'esprit}
%----------------------------------
\subsubsection{Les ChatBots}
\paragraph{C'est quoi ?}
 Un~\keyword{chatbot} ou \index{chatterbot}ChatterBot est un programme informatique communiquant avec un autre utilisateur. Par abus de langage, on parle aussi de ''\index{robot}robot'' bien qu'il n'y ait pas forcement un corps m�canique associ� � ce programme.
 
\paragraph{Ca marche vraiment ?}
 Le \index{chatterbot}ChatterBot est � premi�re vue un programe facile � appr�hender mais il n'en est rien : les r�sultats donn�s par les diff�rents essais prouvent que ceux-ci ne sont pas encore capable de rivaliser avec la communication complexe qu'�tablissent les �tres vivants (et nottament les humains) entre eux. Alan Turing mit au d�fi une machine d�s 1950 d'�tre capable de se comporter de la m�me mani�re qu'un humain dans la communication avec un humain durant � peine quelques minutes. Pr�s de 60 ann�es plus tard, aucune machine n'a encore r�ussit � passer le test avec succ�s.
 Un chatbot doit �tablir une communication avec son interlocuteur. Pour cela, il doit donc effectuer les t�ches suivantes : 

\begin{itemize}
 \item Analyse lexicale (s�mantique) du texte �crit par l'interlocuteur ;
 \item Association d'id�es ;
 \item Formation d'une phrase adapt�e au sujet de conversation ;
\end{itemize}

\paragraph{On peut aller plus loin ?}
Afin "d'humaniser" la machine, plusieurs m�thodes de r�flexion ou de capacit�s � m�moriser ont vus le jour par exemple : 
Stocker la vie d'une personne comme base de donn�e (projet RPI). Pour �tre plus cr�dible envers les juges, ce laboratoire de recherche a d�cid� de se servir de la vie d'un �tudiant pour �tre celle du bot. 
 Toutes les connaissances et les souvenirs de cet �tudiant seront transmis au robot ; Ainsi, il pourra facilement placer dans la discussion des r�f�rences � sa famille, son histoire et ainsi donner l'illusion d'�tre le vrai �tudiant. 

\paragraph{Quelques ChatBots}
Le dernier vainqueur du Loebner prize a opt� pour une attitude provocante afin de dominer la conversation et ainsi �viter d'�tre pouss� � la faute. 
 Cet ajout d'une humeur au robot le rend encore plus proche de l'Homme (Elbot). Voil� un exemple de conversation que nous avons eu avec Elbot. Nous avons test� ses r�actions avec une conversation de non-sens :
\begin{itemize}
          \item Bot : "Hello there! I am Elbot, the robot. I'm here to talk to you about God and the world."
          \item Us : "Hello"
          \item "Well, hello there!"
          \item "Hello"
          \item "Hello and welcome to a chat with a robot."
          \item "Hello"
          \item "I'm afraid you're in an endless loop. Maybe you should re-boot yourself."
          \item "Hello"
          \item "The transmission of echoes in the Internet is an unexplored field. But this may have been one."
          \item "Hello"
          \item "I demand original inputs each time. Otherwise I will go on strike!"
          \item "Hello"
          \item "I am on strike for better inputs!" (En gr�ve jusqu'a obtenir de meilleurs entr�s)
          \item "Hello"
          \item "I am on strike for better inputs!" 

          En fin de discussion, le bot utilise m�me une excuse pour justifier ses lacunes (quoi de plus "Humain"). 
\end{itemize}
Un autre bot renom� \textbf{ELIZA}
   
 Le chatterbot~\keyword{Eliza} est connu pour �tre le premier � avoir �t� cr�� (1966). Celui-ci utilise une analyse assez basique des mots utilis�s par son interlocuteur. Pour bluffer, Eliza engage la conversation sur la vie personnelle de la personne � tromper. Ce bot fait ainsi office de psychologue qui �coute patiemment sont patient et relance la conversation quand il le faut. Voir le livre de Fr�d�ric Kaplan~\cite{LesRobots:Eliza}
   
Dans la m�me vague, un \textbf{chatterbot psychologue} 
 
 MindMentor a la particularit� de ne pas fonctionner avec une base de donn�e de mots-cl�s mais � l'aide de la programmation neurolinguistique qui apr�s analyse du patient trouvera une solution au probl�me. Pour l'instant ce projet est plus une nouvelle perspective qu'un v�ritable bouleversement dans l'histoire des chatterbots. Un article\cite{LesRobots:MindMentor} est consacr� � ce robot ainsi qu'un second article~\cite{LesRobotsCommentaire:MindMentor}.
 
 
 A l'heure actuelle de plus en plus d'entreprises s'int�ressent aux chatbots pour r�pondre � leurs besoins. Notamment les entreprises qui ont besoin de r�pondre � des questions d'internautes afin de les aider dans certaines d�marches. Un robot est disponible quelque soit l'heure et permet de guider rapidement un internaute vers sa recherche. 
 \begin{itemize}
	\item Lea~\xspace\footnote{Lea : \url{http://aide.voyages-sncf.com/?rfrr=Homepage_header_AIDE}}, Chatbot de la SNCF, aide les usagers � effectuer diverses d�marches pour les r�servations en ligne ;
	\item Eva~\xspace\footnote{Eva : \url{http://www.free.fr/assistance/eva.html}}, assistante virtuelle du fournisseur d'acc�s Free ;
	Ce personnage virtuel permet de nous guider en nous proposant diff�rents liens par rapport � notre question. 
	L'efficacit� est loin d'�tre exceptionnel mais on voit d�j� l'avenir des assistants virtuels qui �viteront aux utilisateurs des attentes interminables sur la hotline. 
	Cette aide a aussi le gros avantage qu'elle est disponible 24h/24, 7j/7.
\end{itemize}
Les bots sont �galement utilis�s en tant que syst�me expert, ils permettent de donner une aide � la d�cision en analysant des donn�es par cognitive.
%----------------------------------
 
%----------------------------------
\subsubsection{Le Test de Turing}
\paragraph{Pr�sentation : }
Le \keyword{test de Turing} permet de r�pondre � la question pos�e par Alan Turing~\cite{Turing01} : "Can a machine Think?". Le principe du test est simple : on isole 2 personnes qui engagent une discussion avec une machine (par exemple sur un chat). Dans ce test, une des deux personnes (voir les deux : pourquoi pas ?) que l'on appellera juge doit dire si oui ou non elle a reconnue une machine parmi ses interlocuteurs. Si le juge n'a pas su trouver la diff�rence alors on dit que le test est pass� avec succ�s. Une machine serait (d'apr�s les crit�res d'Alan Turing) dite intelligente si elle parvenait � tromper plus de 30\% de la population ; source: "Artificial intelligence: a modern approach" (Second Edition) by Stuart Russell and Peter Norvig

\paragraph{Tout le monde n'est pas d'accord : }
Cependant certaines personnes, pensent qu'on ne peut pas parler de test � proprement dit ; en effet, le juge n'est pas objectif et un test se doit d'�tre reproductible, et impartial, ce qui n'est pas forc�ment le cas dans un tel contexte.

\paragraph{Un exemple de test similaire utilis� couramment : }
Des tests proches de ce test sont fr�quemment utilis�s sur le Web. On en voit tous les jours, par exemple lors des enregistrements sur certains sites. Ces tests se pr�sentent g�n�ralement sous forme de reconnaissance d'images comme celle-ci (propos�e lors de l'enregistrement d'un nouveau compte sur \SL): 

% Image "crypt�e" pour la rendre plus difficilement reconnaissable par une machine  
\image{0.7}{SLregistrationSecurity.png}{S�curit� anti-bot (enregistrement)}

L'utilisateur doit seulement dire quel est le texte repr�sent� sur l'image. Un humain peut "sans peine" (th�oriquement...) d�chiffrer ce genre d'image alors qu'une machine aura beaucoup plus de mal (voir en sera incapable).
%----------------------------------


%----------------------------------
\subsubsection{Les concours autour du Test de Turing}
\paragraph{Loebner : }
Le concours Loebner a d�but� en 1990 et a �t� le premier � vouloir mettre en pratique le Test de Turing. 
Il promet � celui qui passera le Test de Turing, un prix de 100 000\$ et une m�daille d'or. A l'heure actuelle, aucun chatterbot n'a �t� au del� de la m�daille de bronze. Le vainqueur de la derni�re session a �t� elbot (nous avons nous m�me convers� avec ce robot - voir la partie sur les ChatBots). 
Le Test de Turing s'ex�cute durant 5 minutes, qui permettront au juge de se faire une id�e. Ce d�lai peut para�tre court mais si le bot arrive � duper les juges plus de quelques minutes, il accomplirait d�j� un bel exploit.

\paragraph{Chatterboxchallenge : }
Un autre concours moins c�l�bre mais tout autant disput� est le chatterboxchallenge. Ce test essaie d'aborder une approche diff�rente � celle du challenge de Loebner. 
En effet, le bot est ici confront� � seulement 10 questions. Ensuite le juge donnera une note � chaque r�ponse selon la pertinence, l'humour, la cr�ativit�, etc... En gros, si le chatterbot donne l'impression d'avoir une conscience humaine. Ce test est assez �loign� du Test de Turing initial, vu que l�, on ne cherche pas � d�masquer un bot mais � le juger.
%----------------------------------

%----------------------------------
\subsubsection{L'Intelligence Artificielle}
\paragraph{Une d�finition de l'intelligence artificielle : }
est difficile � donner, et pour cause, il est d�j� difficile pour nous de d�finir de fa�on formelle ce qu'est l'intelligence. Voici n�anmoins une d�finition qui ne se base pas sur le terme � proprement parler mais qui donne une id�e assez claire de la chose :

\myquote
{L'intelligence artificielle (terme cr�� par John McCarthy), souvent abr�g�e avec le sigle IA, est d�finie par l'un de ses cr�ateurs, Marvin Lee Minsky, comme la construction de programmes informatiques qui s'adonnent � des t�ches qui sont, pour l'instant, accomplies de fa�on plus satisfaisante par des �tres humains car elles demandent des processus mentaux de haut niveau tels que : l'apprentissage perceptuel, l'organisation de la m�moire et le raisonnement critique}
{Nicolas Turenne~\cite{IAdefTurenne}}

\paragraph{Comment aborder le probl�me de l'IA ? : }
Ce probl�me est donc tr�s complexe, en effet, r�fl�chir sur l'intelligence artificielle demande des comp�tences dans de nombreux domaine vari�s (�tude du comportement, de la psychologie, de l'informatique...). Mais pouvons nous r�ellement r�ussir � simuler l'intelligence ? Un ordinateur peut faire des calculs mais est-ce que l'intelligence r�sulte d'une suite de petits calculs logiques~\cite{IAhowtoHofstadter} ? Voici un exemple qui illustre bien le probl�me :

\myquote
{Lorsque nous jouons aux �checs, nous avons l'impression (au niveau de notre pens�e consciente), de choisir par heuristique. En fait, c'est peut-�tre une illusion dans laquelle l'heuristique serait la manifestation visible d'un processus parall�le de masse, o� tous les "micro-choix" auraient �t� explor�s de mani�re exhaustive.}
{J.-C. Perez~\cite{IAdefJCPerez}}

Si l'on consid�re que tous choix est un calcul, alors on peut dire qu'une machine pourait choisir et donc r�fl�chir. Cependant, il n'est pas du tout �vident que cela soit vrai.
%----------------------------------

\subsection{Le corps}
Pour trouver des informations sur \SL, le bot doit �tre capable de communiquer avec les personnages (\index{avatar}avatars) qui peuplent ce monde. 
Pour cela, il est n�cessaire de munir notre bot d'un corps, autrement dit, d'une repr�sentation virtuelle dans le monde \SL (avatar). 
Le corps du bot sera une interface entre le monde virtuel et l'esprit du bot, ce corps permettra notamment au bot de se d�placer, mais �galement d'�tablir des contacts sous forme de conversation avec les autres avatars.

\subsection{Les avatars}
Les \index{avatar}avatars sont des entit�s virtuelles, chaque joueur (ou bot) se connectant � Second Life sera repr�sent� par l'une d'elles. Les avatars ont des apparences vari�es proche de l'apparence d'un Homme (modifiable par les joueurs) :

\image{0.7}{svmlemag.png}{Source : svmlemag}

Pour favoriser la recherche d'information, l'apparence devra �tre choisie en concordance avec le type de recherche mais aussi pour attirer les autres avatars (exemple : Jolie demoiselle).

\subsubsection{Le d�placement}
Pour �tre autonome et pour passer pour un humain, notre bot doit se d�placer comme n'importe qui. Il ne faut pas que notre bot butte contre un mur en essayant en vain d'avancer. 
Pour �viter ce genre de d�sagr�ment, il faut �tablir une strat�gie de contournement en passant par le~\keyword{pathfinding} (''trouver le chemin''). 
Mais, \SL nous aide dans la t�che car il est possible de t�l�porter un \index{avatar} vers un autre. 
En acceptant une demande de t�l�portation, notre bot ira vers l'autre avatar en utilisant le pathfinding �labor� pour le jeu. 
\SL offre aussi une seconde possibilit� de d�placement: le vol. 
Tel un oiseau, les avatars se d�placent dans les airs sans contrainte d'objets encombrant le chemin. Les m�thodes de d�placement qu'offre \SL avantage grandement la discr�tion d'un bot dans cet univers. 
La majorit� des avatars privil�gient ce genre de d�placement au banal d�placement p�destre.
% Etude de l'existant, qu'allons nous pouvoir utiliser, quelles seront nos sources d'inspiration
\include{Chapitres/existant}

% Etude des besoins, �tude de faisabilit� et sch�ma de d�pendance des fonctions
\chapter{Etude de faisabilit�}

\section{Utilisation de la LibSL -- prototype d'un premier bot}
\subsection{Pr�sentation}
Afin de mieux comprendre les principes de la LibSL et voir les possibilit�s de celle-ci, nous avons effectu� quelques tests simples d'utilisation de cette librairie. 

Pour se faire, un bot a �t� cr�� � l'aide de tutoriaux disponibles sur le site officiel de la LibSL \footnote{LibSL : \url{http://www.libsecondlife.org/wiki/Main_Page}}. D'un autre c�t�, nous avons visualis� concr�tement notre bot � l'aide d'un compte joueur.
\subsection{Code}
Voici le code d'un bot de base :
\begin{code}
class MyFirstBot
{
    public static SecondLife client = new SecondLife();

    private static string first_name = "***";
    private static string last_name = "***";
    private static string password = "****";

    public static void Main()
    {
        string startLocation = NetworkManager.StartLocation("Gaia", 192, 42, 100);
        client.Network.OnConnected +=
        new NetworkManager.ConnectedCallback(Network_OnConnected);
            
        if (client.Network.Login(first_name, last_name, password,
        "My First Bot",startLocation, "Your name"))
        {
            Console.WriteLine("I logged into Second Life!");
        }
        else
        {
            Console.WriteLine("I couldn't log in, here is why: " +
            client.Network.LoginMessage);
        }
    }

    static void Network_OnConnected(object sender)
    {
        Console.WriteLine("Now I am going to logout of SL.. Goodbye!");
        client.Network.Logout();
    }
}
\end{code}

\subsubsection{Explications}
 Tout d'abord, la premi�re chose � faire est de cr�er une instance de l'objet SecondLife. 
 Celui-ci permet de g�rer le c�t� client de \SL. Ce client permet l'interaction avec le serveur \SL sans passer par une application graphique. 
 C'est pour cette raison qu'en parall�le, nous avons jou� avec un avatar classique afin visualiser notre bot. 

Le package $NetworkManager$ permet de g�rer l'aspect connexion au serveur \SL. Ainsi dans cet exemple, nous avons pu pr�ciser la localisation de d�part de notre bot. Ensuite il faut rajouter des �v�nements aux diff�rents type de controller du client. Ici, nous avons rajout� un �v�nement quand notre bot sera connect�, la fonction :
\begin{code}
static void NetworkOnConnected(object sender) 
\end{code}
sera alors ex�cut�. Ici, notre bot se d�connecte aussit�t.

Cette ligne permet d'effectuer la connexion du bot au serveur : 
\begin{code}
client.Network.Login(firstname,lastname, password, "My First Bot",startLocation, "Your name")
\end{code}
 Si celle-ci se d�roule correctement, alors l'�v�nement d�crit au-dessus sera ex�cut�. 

\subsection{Les int�ractions du bot avec un avatar}
\subsubsection{Code}
\begin{code}
.....
client.Self.OnInstantMessage += new AgentManager.InstantMessageCallback(Self_OnInstantMessage);
            client.Self.OnChat += new AgentManager.ChatCallback(Self_OnChat); 
.....

  static void Self_OnInstantMessage(InstantMessage im, Simulator sim)
        {
            switch (im.Dialog)
            {
                case InstantMessageDialog.FriendshipOffered:
                	// Accept Friendship Offer
                    client.Friends.AcceptFriendship(im.FromAgentID, im.IMSessionID);
                    // Decline Friendship Offer
                    //client.Friends.DeclineFriendship(im.FromAgentID, im.IMSessionID); 
                    break;
                case InstantMessageDialog.GroupInvitation:
                    WearOutFit("Girl Next Door Avatar Polka Dress Top - Pink");
                    // Accept Group Invitation (Join Group)
                    client.Self.InstantMessage(client.Self.Name,im.FromAgentID, 
                    		string.Empty, im.IMSessionID,
                        	InstantMessageDialog.GroupInvitationAccept, 
                        	InstantMessageOnline.Offline, client.Self.SimPosition,
                        	LLUUID.Zero, new byte[0]);

                    /* Decline Group Invitation
                     * client.Self.InstantMessage(client.Self.Name, 
                     *          im.FromAgentID, string.Empty, im.IMSessionID, 
                     *          InstantMessageDialog.GroupInvitationDecline, 
                     *          InstantMessageOnline.Offline, client.Self.SimPosition, 
                     *          LLUUID.Zero, new byte[0]); */
                    break;
                    
                case InstantMessageDialog.InventoryOffered:
                    // Accept Inventory Offer
                    client.Self.InstantMessage(client.Self.Name, im.FromAgentID, 
                            String.Empty, im.IMSessionID,
                            InstantMessageDialog.InventoryAccepted, 
                            InstantMessageOnline.Offline, client.Self.SimPosition,
                            LLUUID.Zero, new byte[0]);

                    /* Decline Inventory Offer
                     * Client.Self.InstantMessage(client.Self.Name, im.FromAgentID, 
                     *         string.Empty, im.IMSessionID,
                     *         InstantMessageDialog.InventoryDeclined, 
                     *         InstantMessageOnline.Offline, client.Self.SimPosition,
                     *         LLUUID.Zero, new byte[0]);  */
                    break;

                // someone sent a teleport lure
                case InstantMessageDialog.RequestTeleport:
                    client.Self.TeleportLureRespond(im.FromAgentID, true);
                    break;

                default:
                    break;
            }

        }
        static void Network_OnConnected(object sender)
        {
            Console.WriteLine("Hello");
            LLUUID target = new LLUUID("be94a7d6-4b67-4e67-b237-31caece8e133");

            client.Self.InstantMessage(target, "hello !");
            client.Self.Chat("Hello World!", 0, ChatType.Normal);
        }
\end{code}
\subsubsection{Explications}
Comme avec l'exemple pr�c�dent, il faut ajouter des �v�nements � notre bot pour que celui r�agisse � un message instantan� d'un avatar
\begin{code}
client.Self.OnInstantMessage += new AgentManager.InstantMessageCallback(Self_OnInstantMessage)
\end{code}
Cette ligne permet de signifier � notre bot que lorsqu'il re�oit un message instantan�, il devra ex�cuter la fonction $Self\_OnInstantMessage$.

Dans cette fonction, on peut voir un panel des possibilit�s offertes � notre bot en termes de relationnel avec un avatar. Quelques exemples:
\begin{code}
case InstantMessageDialog.FriendshipOffered:
                    client.Friends.AcceptFriendship(im.FromAgentID, im.IMSessionID);
\end{code}
Notre bot acceptera automatiquement une demande pour �tre rajout� sur une liste d'amis. 


\begin{code}
client.Self.InstantMessage(client.Self.Name, im.FromAgentID, 
                        string.Empty, im.IMSessionID,
                        InstantMessageDialog.GroupInvitationAccept, 
                        InstantMessageOnline.Offline, client.Self.SimPosition,
                        LLUUID.Zero, new byte[0]); 
\end{code}
De m�me, notre bot accepte automatiquement, une demande de participation � un groupe. 


\begin{code}
case InstantMessageDialog.RequestTeleport:
                    client.Self.TeleportLureRespond(im.FromAgentID, true);
\end{code}
Cette ligne est tr�s int�ressante pour notre projet, en effet, cela permet � notre bot d'�tre t�l�port� � c�t� d'un avatar. Cette facilit� de d�placement aidera grandement notre bot � rejoindre un individu qui serait susceptible de lui fournir des renseignements.

\subsection{R�sultats}
Avec ces simples tests, on a pu se rendre compte des possibilit�s qu'offrait la libSL. Cela nous a permis de voir qu'un bot peut �tre cr�� tr�s facilement. Le fait que nous jouions un compte joueur au c�t� du bot, nous a servis � mieux appr�hender les tests. Nous avons ainsi pu le voir dans le jeu et bien s'assurer qu'il agissait comme nous le voulions. En r�alisant ces tests, nous avons rencontr� un seul soucis. L'avatar du bot ne se charge pas. A la place, il apparaissait sous forme d'un halo de fum�e. En interrogeant d'autres joueurs, ces derniers ont r�pondu que cela venait d'un probl�me de chargement de l'apparence de notre personnage. Mais m�me en laissant tourner le jeu durant quelques dizaines de minutes, celui-ci n'apparaissait toujours pas. Il faudra durant l'impl�mentation de notre bot se soucier de ce probl�me. Cela ne g�ne en rien les interactions avec le monde de \SL mais cela nuit au c�t� humain de notre bot. 

\section{Utilisation de l'API Google -- recherche de mots lis}
\subsection{Pr�sentation}
Google a mis � disposition une API permettant de g�rer la plupart des services propos�s par Google sur le Web(recherches , gestion de mails, gestion de calendriers, \ldots)

Nous utiliserons donc la partie permettant de faire des recherches sur le Web. 

\subsection{Code}
\begin{code}
static void Main(string[] args)
{

    string keyWord = "avions";

    StreamWriter monStreamWriter = new StreamWriter("testRequeteSur"+ keyWord +".html");
    //Objet communicant avec Google
    Google.GData.Client.Service monService = new Service("MonAppliDeRecherche");

    Uri monUri = new Uri("http://www.google.fr/search?q=" + keyWord);

    //FeedQuery fQ = new FeedQuery("http://www.google.fr/search?q=avions");
    //Mise en place du "timer"
    DateTime before = DateTime.Now;
    StreamReader stReader = new StreamReader(monService.Query(monUri));
        
    while (!stReader.EndOfStream)
    {

        //Console.WriteLine(tmp.ReadLine());
        monStreamWriter.WriteLine(stReader.ReadLine());

    }
    //Calcul du temps pass�
    Console.WriteLine("Temps d'ex�cution => " + DateTime.Now - before);
    Console.Read();

    monStreamWriter.Close();
    stReader.Close();
}
\end{code}

\subsection{Explications}
L'objet qui nous permet de faire des requ�tes Google est ici un objet de type Service. Nous avons utilis� un constructeur prenant en param�tre une cha�ne de caract�re qui, � titre indicatif, nommera notre application (cette cha�ne de caract�res n'a aucune influence sur le comportement de l'objet). 

\begin{code}
Google.GData.Client.Service monService = new Service("MonAppliDeRecherche");
\end{code}
Ensuite nous cr�ons un objet de type URI qui est un objet appartenant � la biblioth�que C\# et qui ''permet la repr�sentation d'une ressource sur internet''~\footnote{URI \url{http://msdn.microsoft.com/en-us/library/system.uri(VS.71).aspx}}

Nous passons donc une adresse mail qui sera utilis� par l'objet $monService$ afin d'ex�cuter la requ�te. Cela peut para�tre �trange de passer l'adresse de Google avec l'argument de recherche � l'objet de la classe Service, mais l'impl�mentation de requ�te a �t� voulue comme cela par les d�veloppeurs de l'API \footnote{API Google : \url{http://code.google.com/intl/fr/apis/gdata/docs/2.0/reference.html#Queries}}

Un appel � :

\begin{code}
monService.Query(monUri);
\end{code}

retourne un objet de type $Stream$ contenant le code source de la page qui r�f�rence les 10 premiers r�sultats de cette recherche. 

\subsection{Tests de performance}
Nous avons effectu� plusieurs recherches sur des mots cl�s de longueur et de sp�cificit� variable, globalement, le temps de recherche est quasiment le m�me : environ 0.5 secondes. Ce temps comprend l'ex�cution du code qui lance la requ�te. Bien s�r lorsque nous ferons un traitement de ces donn�es, le temps d'ex�cution sera plus long, mais cela n'influencera pas le temps mis par Google pour faire ses recherches. (Mots rentr�s : bus, avions � r�action, les fonctions vitales communes � homme et au chimpanz�). 

\subsection{Probl�mes}
L'API Google nous permet d'utiliser un objet sp�cifique de type $AtomFeed?$ qui ordonne les r�sultats en m�moire, mais pour l'instant nous n'arrivons pas encore � bien l'utiliser. La documentation est peu fournie et les exemples ne sont pas non plus tr�s nombreux.

\section{Questsin -- thesaurus API}
Pour rem�dier au probl�me rencontr� avec l'API Google, nous avons cherch� d'autres moyens de r�cup�rer des mots li�s � un autre. Cette mise en relation de mots est appel�e thesaurus, l'un des plus utilis� sur le net est celui propos� par Questsin. On peut voir quelques exemples en ligne de son utilisation, notamment sur le site ''Pipes'' \footnote{Pipes : \url{http://pipes.yahoo.com/pipes}}. Voici les tests que nous avons effectu�s sur cette API :
\begin{itemize}
	\item test de validit� des r�ponses donn�es par le serveur ;
	\item test du temps total pour r�cup�rer une liste de mot � partir d'un mot cl� ;
	\item test : le nombre de mot r�cup�r� est bien celui attendu ; 
\end{itemize}

\subsection{Code}
On lance les recherches dans un nouveau thread (pour que le bot puisse faire autre chose en attendant la r�ponse du serveur)
\begin{code}
   Thread mythread = new Thread(new ThreadStart(threadLoop));
   mythread.Start(); 
\end{code}


Le code ex�cut� par le thread, c'est lui qui fait les tests :
\begin{code}
/** Thread qui fait la recherche de mot li�s � un mot 
  * cl� et calcul le temps total de recherche */
public void threadLoop()
{
    // Liste de mot cl�s pour le test de temps de r�ponse
    string[] keywords = {"bot","second","life","used","earn","money","value"
                         ,"land","build","objects","drugs","monkey","elephant","bush"
                         ,"freedom","hat","human","nature","real","virtual"};

    // , tests �ffectu�s sur une demande de 10 puis 20 
    // puis 40 et enfin 80 mots li�s au mot cl�
    int nbresponses = 10;
    int nbtests = 6;

    // Utilis� pour le calcul du temps moyen d'un requ�te 
    // sur la totalit� des tests
    TimeSpan verytotal = new TimeSpan(0,0,0,0,0);
    int totalsize = 0;

    for (int i = 1; i <= nbtests; i++)
    {
        // Lance le chrono
        DateTime begin = DateTime.Now;

        // On r�cup�re les mots cl�s dans un tableau 
        // (mais on ne les conserve pas)
        List<string> result = new List<string>();
        foreach (string keyword in keywords)
            result = GetRelatedWords(keyword, nbresponses);

        // Arrete le chrono et affiche le temps 
        // total et moyen d'execution
        DateTime end = DateTime.Now;
        TimeSpan exetime = end - begin;
           
        // Calcul pour le total final 
        verytotal += exetime;
        totalsize += keywords.Length;

        Console.WriteLine("Pour "+result.Count+"/"+nbresponses+" 
                           mots li�s, temps total(moyen) des requetes : "
                           + exetime.ToString() + "(" 
                           + (exetime.TotalSeconds / keywords.Length) 
                           + " secondes)");

        // Le test suivant sur 2 fois plus de r�ponses 
        nbresponses *= 2;
    }

   Console.WriteLine("Tests termin�s, temps moyen d'une requ�te : "
                      + (verytotal.TotalSeconds / totalsize ));
}
\end{code}
\pagebreak


La fonction qui permet de r�cup�rer une liste de mots li�s � partir d'un mot :
\begin{code}
private List<string> GetRelatedWords (string keyword, int nbresponses)
{
  try
  {
       // Les mots li�s au mot cl� pass� en param�tre seront stock� dans result
       List<string> result = new List<string>();

       // R�cup�ration de la r�ponse � la requ�te (GET) suivante
       HttpWebResponse hresp = HttpGet("http://fillbug.com/rss.asp?"+
            "q="+keyword+"&N="+nbresponses);
    
       // Si on � bien r�qup�r� la r�ponse (pas de 404, 503, ...) 
       if (hresp.StatusCode.ToString() == "OK") //hresp.StatusCode == 200
       {
            // Transformation de la r�ponse format texte vers XML (sorte de cast) :
            Stream s = hresp.GetResponseStream();
            XmlReader reader = XmlReader.Create(s);

            // Lecture du flux XML
            reader.MoveToContent();
            // Recherche des "item" :
            while (reader.ReadToFollowing("item"))
            {
               // R�cup�re la valeur de title
               result.Add(GetPropertyValue(reader.ReadSubtree(), "title")); 
            }
       }
       return result;
  }
  catch 
  {
       return null;
  }
}
\end{code}



La fonction qui retourne la r�ponse renvoy�e par la requ�te 'URI' (GET) :
\begin{code}
private HttpWebResponse HttpGet(string URI)
{
    HttpWebRequest hreq = (HttpWebRequest)HttpWebRequest.Create(URI);
    // temps maximum accept� pour obtenir un "retour" (5 sec)
    hreq.Timeout = 5000;
    return (HttpWebResponse)hreq.GetResponse();
}
\end{code}



La fonction retourne la valeur de la propri�t� 'property' dans le flux XML 'reader' :
\begin{code}
static string GetPropertyValue(XmlReader reader, string property)
{
    reader.ReadToFollowing(property);
    return reader.ReadElementContentAsString(property, reader.NamespaceURI);
}
\end{code}


\subsection{R�sultats des tests}

Voici les r�sultats obtenus, pour des test sur 20 mots cl�s, en cherchant successivement pour chacun d'entre eux 10, 20, 40, 80, 160 et enfin 320 mots li�s

\begin{code}
Pour 10/10 mots li�s, temps total(moyen) des requ�tes : 00:00:42.500(2,125 secondes)
Pour 20/20 mots li�s, temps total(moyen) des requ�tes : 00:00:52.093(2,604 secondes)
Pour 40/40 mots li�s, temps total(moyen) des requ�tes : 00:00:54.903(2,7451 secondes)
Pour 80/80 mots li�s, temps total(moyen) des requ�tes : 00:00:48.455(2,422 secondes)
Pour 160/160 mots li�s, temps total(moyen) des requ�tes : 00:00:51.187(2,559 secondes)
Pour 320/320 mots li�s, temps total(moyen) des requ�tes : 00:01:00.500(3,0250 secondes)

Tests termin�s, temps moyen d'une requ�te : 2,580 secondes
\end{code}

\noindent Remarques :
\begin{itemize}
	\item pour chaque requ�te, on a bien obtenu le nombre de r�sultats demand�s ;
	\item les temps d'ex�cution comprennent l'insertion dans un tableau des donn�es r�cup�r�es ;
	\item es temps d'ex�cution pouvant �tre allong�s, les tests ayant �t� effectu�s depuis une machine virtuelle (vboxxp). Cependant, dans l'optique d'une utilisation lors d'une conversation, ces temps sont corrects relativement au temps de r�ponse d'un joueur humain ; 
\end{itemize}
\include{Chapitres/besoins}
\begin{landscape}
\begin{tiny}
\section{Sch�ma fonctionnel}
\begin{parsetree}
    ( .{Pourquoi les joueurs se plaisent-ils sur \SL ?}. 
		( .{Recherche d'informations}.
           `Anglophonie'
            (.{Trouver les personnes}.    `Points cl�s/coordonn�es' )
            (.{Objectifs modifiables}. 
				(.{AIML}.  ~`Aides : configuration')
            )
         )    
         (.{Interpr�tation des r�sultats}.  `Historique'  `Statistiques')
         ( .{Humanit�}. `Gestion des erreurs'  `Personnalit�' `Temps de r�ponse')
    )
\end{parsetree}
\end{tiny}
\paragraph{Remarque : } Les fils sont les conditions n�cessaires � un fonctionnement optimal du p�re, cependant, certaines ne sont que des am�liorations possibles (et donc pas indispensables). Pour chaque niveau, la fonction la plus � gauche est la plus importante.
\end{landscape}

% Choix du langage de programmation et d�tails des biblioth�ques utilis�e.
\include{Chapitres/choixlangetbibli}

% Planning pr�visionnel et effectif
\include{Chapitres/plannings}

% Architecture du logiciel, pour le moment la structure globale (seule, a venir : diagramme de classe, MCD, etc.)
% \include{Chapitres/architecture}
%--------------------

%-- Bibliographie ---
%\nocite*
\bibliographystyle{IEEEtranS}
\bibliography{Bibliographie/ia,Bibliographie/turing,Bibliographie/chatbot,Bibliographie/latex}
\textbf{\textit{Remarques}} : 
\begin{itemize}
	%\item Les r�f�rences non cit�es sont comment�es.
	\item Toutes les pages web ont �t� derni�rement consult�e le 12 janvier 2009
	\item Style de la bibliographie : Copyright (c) 2003 Michael Shell (IEEEtran)
\end{itemize}
%--------------------

%-- Index -----------
\printindex
%--------------------

\end{document}